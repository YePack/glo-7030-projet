Computer vision is a growing field that changed the faces of many applications in robotic, security or even health sectors. Another field that is currently having more and more interests in such computer vision applications is the sports analytics. The professionnal sports clubs are using analytics and data to understand as more as they can the game and the performances of their players and their opponents. To increase that understanding, you often need a bunch of experts analyzing a bunch of data. Computer vision is well suited to extract that data because it allows the detection of many events simultaneously, which otherwise may have been done by many humans. In order to extract that data properly, it's way more interesting to map those events on the field (or the ice). To do so, it's often necessary to understand the general representation of the moment (or the image), which could be done by a computer vision task called semantic segmentation.
\\
In his work, \cite{Homayounfar} present a methodology where he uses different cues on the field such as lines, corners, circles and so on to train another model that position the field in a 2 dimensionnal plan. In our project, we tried to improve the cues detection technic by training different semantic segmentation models and gain insights on how to train such models. The next step following that project will be to also use that representation in order to map players into a 2 dimensionnal plan.
\\
In the next section, we will start by giving some background about the task of semantic segmentation (section \ref{sec:background}), then we'll present our methodology (section \ref{sec:methodology}), the dataset we used for our experiments (section \ref{sec:dataset}), the results we had (section \ref{sec:results}) and finally a short conclusion (section \ref{sec:conclusion}).
 