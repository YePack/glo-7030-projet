\documentclass[25pt, a0paper,
               colspace=15mm, subcolspace=0mm,
               blockverticalspace=17mm]{tikzposter} % See Section 

\usepackage{graal-poster}
\usepackage{array}
\usepackage{multirow}
\usepackage{multicol}


\definecolor{PaleBlue}{rgb}{0,.55,.9}
\definecolor{PaleGreen}{rgb}{0,.7,.25}
\definecolor{RedPink}{rgb}{.9,0,.2}
\definecolor{Pink}{rgb}{.85,.35,.7}
\definecolor{Purple}{rgb}{.6,0,.75}
\definecolor{Orange}{rgb}{.9,.3,.05}

\colorlet{attentionColor}{Orange}
\colorlet{charEmbedColor}{RedPink}
\colorlet{predEmbedColor}{Pink}
% \colorlet{attentionColor}{GoldUL!90!black}
% \definecolor{attentionColor}{rgb}{.85,.5,.6}



\def\pathwidth{2pt}
\def\nodewidth{3pt}
\def\cornerCurvature{7pt}

\tikzstyle{embed}=[%
  draw,
  #1,
  % line width=3pt,
  anchor=north,
  minimum width=.8cm,
  minimum height=1.6cm,
  inner sep=0pt,
  text=#1!65!black,
  font=\fontsize{25pt}{24}\selectfont,
  ]




\title{\parbox{\linewidth}{\centering Predicting and interpreting embeddings for \\ out of vocabulary words in downstream tasks}}
\institute{Department of Computer Science and Software Engineering, Université Laval}
\author{Nicolas Garneau\up{\dag}, Jean-Samuel Leboeuf\hspace{5pt}\up{\dag}, Luc Lamontagne}

\begin{document}
\maketitle

\begin{columns}
\column{.4}
\block{Introduction}{%
We propose a novel way to handle out of vocabulary (OOV) words in downstream natural language processing (NLP) tasks. We implement a network that predicts useful embeddings for OOV words based on their morphology and the context in which they appear.

\vspace{5mm}
\textbf{Motivations:}
\begin{itemize}
  \item OOV words handling in NLP task is an \colorbold{underestimated problem}.
  % \item Goldberg (2017) emphasizes this fact for NLP tasks such as part of speech tagging (POS) or named entity recognition (NER).
  \item Few learned, end-to-end, solutions proposed.
\end{itemize}

\vspace{5mm}
\textbf{Related work:}
\begin{itemize}
  \item Pinter et al. (2017): Predict OOV embedding using the characters.
  \item Bahdanau et al. (2017): Learn OOV representation from their definition in a dictionary.
\end{itemize}

\vspace{0mm}
\textbf{Goals:}
\begin{itemize}
  \item Evaluate the impact of OOV words in labeling tasks.
  \item Provide a more meaningful way to handle OOV words using \colorbold{context} and \colorbold{morphology}.
  \item Understand when it is important and what is relevant to model OOV embeddings.
  \item \colorbold{Interpret the predicted embeddings} according to the surrounding linguistic elements.
  \item Provide a ``drop-in'', ``end-to-end'' module.
\end{itemize}

% \vspace{-15pt}

}



























\column{0.6}
\block{OOV handling net}{

\vspace{-15pt}
% Comick v3.0 (TheFinalComick)
\begin{center}
\begin{tikzpicture}[
          scale=1.5,
          lstm_right/.style={draw, minimum width=1cm, minimum height=.5cm, inner sep=0pt, anchor=north, PaleGreen},
          lstm_left/.style={draw, minimum width=1cm, minimum height=.5cm, inner sep=0pt, PaleBlue},
          every path/.style={line width=\pathwidth},
          every node/.append style={line width=\nodewidth}]
  \def\w{1cm}
  \def\h{2cm}
  \def\y{-.2}
  \def\scalespacetoken{1.5}
  \def\scalespacecontext{2}
  \def\oplstmdist{-1}
  \def\wordToEmbed{-5mm}
  \def\embedToLstm{-7mm}
  \def\whiteLineWidth{15pt}

  % \draw (-6,\y) -- (10,\y);
  % Char embedding
  \foreach \letter [count=\i] in {S, a, l, e, m}{
    \pgfmathsetmacro\pos{\i-1}
    \node[anchor=mid, inner sep=0](\letter) at (\scalespacetoken*\pos,0) {\letter};
    \node[embed=charEmbedColor](\letter _embed) at (\scalespacetoken*\pos,\wordToEmbed){c};
    \node[lstm_left](\letter _lstm_left) at ([yshift=\embedToLstm]\letter _embed.south) {};
    \node[lstm_right](\letter _lstm_right) at (\letter _lstm_left.south) {};

    \draw[-mytip] (\letter _embed) -- (\letter _lstm_left);
  };

  \draw[decorate,decoration={brace,amplitude=.6cm}]
    ([yshift=1pt,xshift=-.15cm]S.north west) -- node[midway, yshift=12mm]{Unknown word} ([yshift=1pt,xshift=.15cm]m.north east);

  % LSTM arrows
  \foreach \a/\b in {S/a, a/l, l/e, e/m}{
    \draw[mytip-, PaleBlue] (\a _lstm_left) -- (\b _lstm_left);
    \draw[-mytip, PaleGreen] (\a _lstm_right) -- (\b _lstm_right);
  };

  \node[operator](cat_word) at ([yshift=\oplstmdist cm]l_lstm_right.south) {\large $\|$};
  \draw[-mytip, rounded corners=\cornerCurvature, PaleBlue] (S_lstm_left.west) -| +(-.5,-.65) |- (cat_word);
  \draw[-mytip, rounded corners=\cornerCurvature, PaleGreen] (m_lstm_right.east) -| +(.3,-.43) |- (cat_word);

  % Left side
  \def\leftcontdist{-.5*\scalespacecontext}
  \foreach \word [count=\i] in {goalkeeper, which, goal, Syrian}{
    \pgfmathsetmacro\pos{-\scalespacecontext*\i+\leftcontdist}
    \node[anchor=mid, inner sep=0, baseline=0](\word) at (\pos,0) {\word};
    \node[embed=Purple](\word _embed) at (\pos,\wordToEmbed){w};
    \node[lstm_left](\word _lstm_left) at ([yshift=\embedToLstm]\word _embed.south) {};
    \node[lstm_right](\word _lstm_right) at (\word _lstm_left.south) {};

    \draw[-mytip] (\word _embed) -- (\word _lstm_left);
  };

  \foreach \a/\b in {Syrian/goal, goal/which, which/goalkeeper}{
    \draw[mytip-, PaleBlue] (\a _lstm_left) -- (\b _lstm_left);
    \draw[-mytip, PaleGreen] (\a _lstm_right) -- (\b _lstm_right);
  };

  \def\opdist{-2.2}

  \path (Syrian_lstm_right.south west) -- (goalkeeper_lstm_right.south east) node[midway, operator, yshift=\opdist cm](cat_left) {\large $\|$};

  \draw[-mytip, rounded corners=\cornerCurvature, PaleBlue] (Syrian_lstm_left.west) -| +(-.5,-.65) |- (cat_left);
  \draw[-mytip, rounded corners=\cornerCurvature, PaleGreen] (goalkeeper_lstm_right.east) -| +(.3,-.43) |- (cat_left);
  
  \draw[decorate,decoration={brace,amplitude=.6cm}]
    ([yshift=1pt,xshift=-.15cm]Syrian.north west) -- node[midway, yshift=12mm]{Left context} ([yshift=1pt,xshift=.15cm]goalkeeper.north east);

  % Right side
  \def\rightcontdist{3.5*\scalespacecontext}
  \foreach \word/\tag [count=\i] in {Bitar, appeared, to, have}{
    \pgfmathsetmacro\pos{\scalespacecontext*\i+\rightcontdist}
    \node[anchor=mid, inner sep=0, baseline=0](\word) at (\pos,0) {\word};
    \node[embed=Purple](\word _embed) at (\pos,\wordToEmbed){w};
    \node[lstm_left](\word _lstm_left) at ([yshift=\embedToLstm]\word _embed.south) {};
    \node[lstm_right](\word _lstm_right) at (\word _lstm_left.south) {};

    \draw[-mytip] (\word _embed) -- (\word _lstm_left);
  };

  \foreach \a/\b in {Bitar/appeared, appeared/to, to/have}{
    \draw[mytip-, PaleBlue] (\a _lstm_left) -- (\b _lstm_left);
    \draw[-mytip, PaleGreen] (\a _lstm_right) -- (\b _lstm_right);
  };

  \draw[decorate,decoration={brace,amplitude=.6cm}]
    ([yshift=1pt,xshift=-.15cm]Bitar.north west) -- node[midway, yshift=12mm]{Right context} ([yshift=1pt,xshift=.15cm]have.north east);

  \def\decalage{-.0}
  \pgfmathsetmacro\opdistright{\opdist + \decalage}
  \path (Bitar_lstm_right.south west) -- (have_lstm_right.south east) node[midway, operator, yshift=\opdistright cm](cat_right) {\large $\|$};

  \coordinate(mid) at ([yshift=.38cm]cat_right.north);
  \draw[-mytip, rounded corners=\cornerCurvature, PaleGreen] (have_lstm_right.east) -| ++(.3,-.25) %node[fill]{}
  |- ([yshift=-.15cm]mid) %node[fill]{}
  -| ++(-2,-.3) %node[fill]{}
  |- (cat_right);

  \path (Bitar_lstm_left.west)
  -| ++(-.5,-.33) coordinate(p1)
  |- ([yshift=.15cm]mid) coordinate(p2)
  -| ++(2,-.4) coordinate(p3)
  |- (cat_right);
  \draw[white, line width=\whiteLineWidth] (p2) -| (p3);
  \draw[-mytip, rounded corners=\cornerCurvature, PaleBlue] (Bitar_lstm_left.west)
  -| (p1)
  |- (p2)
  -| (p3)
  |- (cat_right);
  

  % Head
  \def\fcdist{-.5}
  \pgfmathsetmacro\fcdistleft{\decalage + \fcdist}
  % FCs
  \draw[-mytip] (cat_left.south) -- ++(0,\fcdistleft) node[draw, anchor=north, rectangle split, rectangle split parts=2](fc_left) {Fully connected \nodepart{second} tanh};
  \draw[-mytip] (cat_right.south) -- ++(0,\fcdist) node[draw, anchor=north, rectangle split, rectangle split parts=2](fc_right) {Fully connected \nodepart{second} tanh};
  \draw[-mytip] (cat_word.south) -- ++(0,\fcdist) node[draw, anchor=north, rectangle split, rectangle split parts=2](fc_word) {Fully connected \nodepart{second} tanh};

  % Attention module
  \path ([yshift=2*\fcdist cm]fc_word.south west) -| node[pos=.25, operator, anchor=north, attentionColor](cat_att) {\Large $\|$} (fc_left.east);
  \draw[-mytip, rounded corners=\cornerCurvature, attentionColor] (fc_word.south) |- ++(2*\fcdist,1*\fcdist) -| (cat_att.north);
  \draw[-mytip, attentionColor] (cat_att.south) -- ++(0,\fcdist) node[draw, anchor=north, rectangle split, rectangle split parts=2, text=attentionColor!60!black](fc_att) {Fully connected \nodepart{second} softmax};

  \draw[-mytip, rounded corners=\cornerCurvature, attentionColor] (fc_left.south) |- (cat_att);
  \draw[-mytip, rounded corners=\cornerCurvature, attentionColor] (fc_right.south) |- (cat_att);

  \def\softmaxShift{-.5}
  % Ponderation
  \path ([yshift=\softmaxShift cm]fc_att.west) -| node[attentionColor, midway, operator](pond_left) {\Large $\times$} (fc_left);
  \draw[-mytip, attentionColor] ([yshift=\softmaxShift cm]fc_att.west) -- (pond_left);
  \draw[-mytip] (fc_left) -- (pond_left);
  
  \path ([yshift=\softmaxShift cm]fc_att.east) -| node[attentionColor, midway, operator](pond_right) {\Large $\times$} (fc_right);
  \draw[-mytip, attentionColor] ([yshift=\softmaxShift cm]fc_att.east) -- (pond_right);
  \draw[-mytip] (fc_right) -- (pond_right);

  \path ([yshift=1.5*\fcdist cm]fc_att.south) -| node[attentionColor, midway, operator](pond_word) {\Large $\times$} (fc_word);
  \draw[-mytip, attentionColor, rounded corners=\cornerCurvature] (fc_att.south) |- (pond_word);

  \draw[white, line width=\whiteLineWidth] ([yshift=2*\fcdist cm]fc_word.south) -- ([yshift=.5*\fcdist cm]pond_word);
  \draw[-mytip] (fc_word) -- (pond_word);


  % End
  \draw[-mytip] (pond_word.south) -- ++(0,\fcdist) node[anchor=north, operator](plus) {\large $+$};
  \draw[-mytip, rounded corners=\cornerCurvature] (pond_left) |- (plus);
  \draw[-mytip, rounded corners=\cornerCurvature] (pond_right) |- (plus);


  \node[draw, anchor=north](tanh-fc1) at ([yshift=\fcdist cm]plus.south) {Fully connected};

  \node[draw, predEmbedColor, text=predEmbedColor!40!black, anchor=north] (output) at ([yshift=\fcdist cm]tanh-fc1.south) {Predicted embedding};
  \draw[-mytip] (plus) -- (tanh-fc1);
  \draw[-mytip] (tanh-fc1) -- (output);


  % Legend
  \begin{scope}[xshift=14cm, yshift=-9.5cm,
          every path/.style={line width=\pathwidth},
          every node/.append style={line width=\nodewidth}]
  \def\legspace{-1.3}
  \def\h{1.3cm}
  % First col
  \node[embed=charEmbedColor, anchor=center, minimum height=\h, minimum width=.8cm](leg char embed) at (0,0*\legspace) {c};
  \node[anchor=west] at (.5,0*\legspace) {Character embedding};

  \node[embed=Purple, anchor=center, minimum height=\h, minimum width=.8cm](leg word embed) at (0,1*\legspace) {w};
  \node[anchor=west] at (.5,1*\legspace) {Word embedding};

  \node[draw, PaleBlue, minimum width=1cm, inner sep=0pt, minimum height=.5cm, anchor=south] (leg lstm left) at (0,2*\legspace) {};
  \node[draw, PaleGreen, minimum width=1cm, inner sep=0pt, minimum height=.5cm, anchor=north] (leg lstm right) at (leg lstm left.south) {};
  \node[anchor=west] at (.5,2*\legspace) {Bidirectional LSTM};

  \node[operator, minimum size=14mm](leg cat) at (0,3*\legspace) {$\|$};
  \node[anchor=west] at (.5,3*\legspace) {Concatenation};

  \node[operator, minimum size=14mm](leg plus) at (0,4*\legspace) {$+$};
  \node[anchor=west, align=left] at (.5,4*\legspace) {Element-wise\\[-3pt] summation};

  \node[operator, minimum size=14mm](leg mult) at (0,5*\legspace) {$\times$};
  \node[anchor=west, align=left] at (.5,5*\legspace) {Multiplication\\[-3pt] by a constant};

  \draw[attentionColor, line width=4pt] (-.35, 6*\legspace) -- (.35, 6*\legspace);
  \node[anchor=west, align=left] at (.5, 6*\legspace) {Attention module};
  \end{scope}

\end{tikzpicture}
\end{center}

The net consists in 3 bi-LSTM taking as input the left context, the right context and the word characters. An attention module ponderates their outputs which are then combined in a last fully connected layer.

}
\end{columns}











\begin{columns}

\column{.2}


% \block[bodyoffsety=48mm, titleoffsety=48mm]{Experiments}{
\block[bodyoffsety=37mm, titleoffsety=37mm]{Experiments}{

\textbf{Set up:}
\begin{itemize}
  \item Labeling tasks:
  \begin{itemize}
      \item \colorbold{Named Entity Recognition} (NER).
      \item \colorbold{POS tagging} (POS).
  \end{itemize}
  \item Dataset: \colorbold{CoNLL 2003}
  
%   \item Two nets working together:
%   \begin{itemize}
%     \item One predicts OOV embeddings (see OOV handling net section).
%     \item One predicts tags (see Labeling task section).
%   \end{itemize}
  % \item Baseline: randomly generated embeddings for OOV.
\end{itemize}


\textbf{Training details:}
\begin{itemize}
    \item Tensors sizes:
    \begin{itemize}
        \item Char. emb.: 20.
        \item Word emb.: 100 (\colorbold{GloVe}).
        \item LSTMs hidden state: 128.
    \end{itemize}
    \item Context size from 2 words to the whole sentence.
    \item Standard learning rate on the labeling task parameters, reduced learning rate on Comick using SGD (0.01, 0.001).
\end{itemize}


% \textbf{Experiments:}
% \begin{itemize}
%   \item Performance gains:
%   \begin{itemize}
%     \item Comparison between accuracies for POS.
%     \item Comparison between F1 scores for NER.
%   \end{itemize}
%   \item Interpretability:
%   \begin{itemize}
%     \item Comparison of the average weights given to the left context, right context and the words by the network by tags.
%     \item Qualitative examples of where we observe a shift of attention according to the context.
%   \end{itemize}
% \end{itemize}

\vspace{-.25mm}

}


  \column{.8}
  \block[bodyoffsety=0mm, titleoffsety=0mm]{Examples}{
  \vspace{-2mm}
  \begin{center}
  \Large
  % \resizebox{\textwidth}{!}{%
  \setlength{\tabcolsep}{8pt}
  \begin{tabular}{c c c c c c c c}
  \toprule
  \multirow{2}{*}{\textbf{Entity}} & \multicolumn{3}{c}{\textbf{Ponderation}} & \multirow{2}{*}{\textbf{Examples}}\\
  \cline{2-4}
  \addlinespace[2mm]
  & Word & Left & Right & \\
  \midrule
  \texttt{PER} & 0.19 &  \colorbold{0.49} & 0.32 & \textbf{in sentencing darrel} \underline{\textit{voeks}} , 38 , to a 10-year prison term on thursday\\
  \texttt{PER} & 0.15 &  \colorbold{0.59} & 0.26 &   \textbf{\texttt{<BOS>} australian parliamentarian john} \underline{\textit{langmore}} has formally resigned from his lower house\\
  \texttt{PER} & 0.15 &  \colorbold{0.61} & 0.24 &  had received today \textbf{from mr john vance} \underline{\textit{langmore}} , a letter resigning his place as \\
  \texttt{PER} & 0.15 &  \colorbold{0.69} & 0.16 &  \textbf{\texttt{<BOS>} rtrs - australian mp john} \underline{\textit{langmore}} formally resigns . \texttt{<EOS>}\\
  \texttt{ORG} & 0.22 & \colorbold{0.46} & 0.32 &  the number of plastic surgeries in [...] the \textbf{brazilian plastic surgery society} ( \underline{\textit{sbcp}} ) , said ,\\
  \texttt{ORG} & 0.28 & 0.23 & \colorbold{0.49} &  to increase them in the united states , " \underline{\textit{sbcp}} \textbf{vice-president oswaldo saldanha said}\\
  \texttt{LOC} & 0.16 & 0.22 & \colorbold{0.62} &  some residents of the \underline{\textit{kazanluk}} \textbf{area are moslems who} converted to islam during \\
  \texttt{LOC} & 0.20 & \colorbold{0.47} & 0.33 &  at a mosque in the \textbf{central bulgarian town of} \underline{\textit{kazanluk}} , causing damage but no injuries\\
  \texttt{MISC} & \colorbold{0.68} & 0.11 & 0.21 &  freestyle \textbf{\underline{\textit{skiing-world}}} cup aerials results .\\
  \texttt{MISC} & \colorbold{0.42} & 0.18 & \colorbold{0.40} & the \textbf{\underline{\textit{franco-african}} summit decided to send} a mission bangui [...] civil war .\\
  %0.17 &  0.40 & \colorbold{0.43} &   \textbf{\texttt{<BOS>}} \textit{langmore} \textbf{, 57 , announced in november that}\\
  \bottomrule
  \end{tabular}
  \end{center}
  
  \vspace{2mm}
  Qualitative example on several OOV words (underlined). We can see that depending on the context and the target, the weights may shift drastically.
  % \vspace{5.5mm}
  }
  




\end{columns}




















\begin{columns}

  \column{.3}


  \block{Labeling task net}{

  \begin{center}
  \begin{tikzpicture}[
          scale=1.5,
          lstm_right/.style={draw, minimum width=1cm, minimum height=.5cm, inner sep=0pt, anchor=north, PaleGreen},
          lstm_left/.style={draw, minimum width=1cm, minimum height=.5cm, inner sep=0pt, PaleBlue},
          every path/.style={line width=\pathwidth},
          every node/.append style={line width=\nodewidth}]
    \def\h{2cm}
    \def\scalespace{2.2}
    \def\w{145mm}
    \def\arrowLen{.6}
    \def\descSpace{-.2}

    % Char embedding

    \foreach \word/\color/\w [count=\i] in {Salem/gray/o, Bitar/gray/o, appeared/Purple/w, to/Purple/w, have/Purple/w}{
      \pgfmathsetmacro\pos{\i-1}
      \node[anchor=mid, inner sep=0](\word) at (\scalespace*\pos,0) {\word};
      \node[embed=\color,draw=\color](\word _embed) at (\scalespace*\pos,-\arrowLen){\w};
    };

    \node[embed=black, minimum width=\w, minimum height=\h, anchor=north](comick) at ([yshift=-\arrowLen cm]appeared_embed.south) {OOV handling net};
    % \node[anchor=east, align=left] at ([xshift=\descSpace cm]comick.west) {OOV \\ handling};

    \foreach \word/\color/\w [count=\i] in {Salem/predEmbedColor/p, Bitar/predEmbedColor/p, appeared/Purple/w, to/Purple/w, have/Purple/w}{
      \pgfmathsetmacro\pos{\i-1}

      \path (\word _embed) edge[draw, -mytip] (comick.north -| \word _embed.south);
      \draw[-mytip] (comick.south -| \word _embed) -- ++(0,-\arrowLen) node[anchor=north, embed=\color, draw=\color](\word _pred_embed) {\w};

      \node[lstm_left, draw=PaleBlue, anchor=north](\word _lstm_left) at ([yshift=-\arrowLen cm]\word _pred_embed.south) {};
      \node[lstm_right, draw=PaleGreen](\word _lstm_right) at (\word _lstm_left.south) {};
      \draw[-mytip] (\word _pred_embed) -- (\word _lstm_left);
  
    };

    % LSTM arrows
    \foreach \a/\b in {Salem/Bitar, Bitar/appeared, appeared/to, to/have}{
      \draw[mytip-, PaleBlue] (\a _lstm_left) -- (\b _lstm_left);
      \draw[-mytip, PaleGreen] (\a _lstm_right) -- (\b _lstm_right);
    };
    % \node[anchor=east, align=left] at ([xshift=\descSpace cm]Salem_lstm_left.south west) {Bi-LSTM \\ task net};


    \foreach \word/\tag [count=\i] in {Salem/B-PER, Bitar/I-PER, appeared/O, to/O, have/O}{
      \pgfmathsetmacro\pos{\i-1}

      \draw[-mytip] (\word  _lstm_right.south) -- ++(0,-\arrowLen) node[embed=Orange](\word _softmax) {s};

      \node[anchor=mid, inner sep=0] (\word _tag) at ([yshift=-.5cm]\word _softmax.south) {\tag};
    };
    % \node[anchor=east, align=right] at ([xshift=\descSpace cm]Salem_tag.west) {Tags};


    % Legend
    % \begin{scope}[xshift=10.7cm, yshift=-2cm,
    \begin{scope}[xshift=-15mm, yshift=-105mm,
            every path/.style={line width=\pathwidth},
            every node/.append style={line width=\nodewidth}]
    \def\legspace{-1.3}
    \def\colsep{7}
    \def\h{1.3cm}
    % First col
    \node[embed=gray, anchor=center, minimum height=\h, minimum width=.8cm](leg char embed) at (0,0*\legspace) {o};
    \node[anchor=west] at (.5,0*\legspace) {OOV word embedding};

    \node[embed=Purple, anchor=center, minimum height=\h, minimum width=.8cm](leg word embed) at (0,1*\legspace) {w};
    \node[anchor=west] at (.5,1*\legspace) {Word embedding};

    \node[embed=predEmbedColor, anchor=center, minimum height=\h, minimum width=.8cm](leg word embed) at (0,2*\legspace) {p};
    \node[anchor=west, align=left] at (.5,2*\legspace) {Predicted embedding};


    \node[draw, PaleBlue, minimum width=1cm, inner sep=0pt, minimum height=.5cm, anchor=south] (leg lstm left) at (\colsep,0*\legspace) {};
    \node[draw, PaleGreen, minimum width=1cm, inner sep=0pt, minimum height=.5cm, anchor=north] (leg lstm right) at (leg lstm left.south) {};
    \node[anchor=west] at (\colsep+.5,0*\legspace) {Bidirectional LSTM};

    \node[embed=Orange, anchor=center, minimum height=\h, minimum width=.8cm](leg word embed) at (\colsep,1*\legspace) {s};
    \node[anchor=west, align=left] at (\colsep+.5,1*\legspace) {Tag prediction\\ with a softmax};

    \end{scope}


  \end{tikzpicture}
  \end{center}
  % \vspace{3mm}

  Two nets working together: the first predicts OOV embeddings (see OOV handling net section) and the second one predicts tags.

  The simple architecture of the labeling net is used to emphasize the usefulness of our module, and to minimize the influence of other factors.

  \vspace{-.5mm}

  }









  \column{.3}
  \block{Interpretability}
  {
  \begin{center}
  \setlength{\tabcolsep}{5mm}
  \begin{tabular}{c c c c c c}
  \toprule
  \multirow{2}{*}{\textbf{Task}} & \multirow{2}{*}{\textbf{Tag}} & \multirow{2}{*}{\textbf{Ex.}} & \multicolumn{3}{c}{\textbf{Ponderation}}\\
  \cline{4-6}
  \addlinespace[3mm]
  & & & Word & Left & Right\\
  \midrule
  \multirow{9}{*}{NER} & O           & 1039  & \colorbold{0.81}    & 0.08    & 0.11 \\
  & B-PERS      & 63    & 0.21    & 0.31    & \colorbold{0.49} \\
  & I-PER     & 119   & 0.16  & \colorbold{0.52} & 0.32 \\
  & B-ORG     & 40  & 0.26  & 0.30  & \colorbold{0.44} \\ 
  & I-ORG     & 3     & 0.27  & 0.31      & \colorbold{0.42} \\
  & B-LOC     & 13  & 0.23      & 0.30  & \colorbold{0.47} \\
  & I-LOC     & 2     & 0.16  & \colorbold{0.48} & 0.36 \\
  & B-MISC      & 47  & \colorbold{0.40} & 0.21  & 0.39 \\
  & I-MISC      & 5     & \colorbold{0.41} & 0.26  & 0.33 \\
  \midrule
  \multirow{5}{*}{POS} & NNP  & 308 & 0.29  & 0.31  & \colorbold{0.40} \\
  & NN  & 46  & \colorbold{0.45} & 0.20  & 0.35 \\
  & CD  & 827 & \colorbold{0.86} & 0.05  & 0.09 \\
  & NNS & 23  & \colorbold{0.37} & 0.24  & \colorbold{0.39} \\
  & JJ  & 100 & \colorbold{0.49} & 0.15  & 0.36 \\
  \bottomrule
  \end{tabular}
  \end{center}
  
  \vspace{1.5mm}
   Average weights assigned to word's characters, left context and right context by the attention mechanism. We can clearly see the shift of attention according to the target entity. We also observe that the attention depends on the task at hand.
  \vspace{-3mm}
  }







  
  
  
  \column{.4}
  

  \block{Performance gain}{%
  \begin{center}
  \setlength{\tabcolsep}{5mm}
  \begin{tabular}{c c c c c}
  \toprule
  \textbf{Task} & \textbf{Metric} & \textbf{Random Emb.} & \textbf{Our module} & \textbf{Gain}\\
  \midrule
  NER      & F1   & 77.56 & \colorbold{80.62} & 3.9\% \\
  POS      & acc. & 91.41 & \colorbold{92.58} & 1.2\% \\
  % Chunking & acc. & 92.63 & 93.16  & \textbf{93.19} \\
  % Keyphrase& F1   & 37.40 & 39.56 & \textbf{39.77} \\
  \bottomrule
  \end{tabular}
  \end{center}
  
  \vspace{2.5mm}
  The impact of our model on two NLP downstream tasks. We compare our OOV embeddings prediction scheme against random embeddings.
  \vspace{-12mm}
  }


  
  
  
  
  
  
  
  
  
  
  
  
  \block{Conclusion}{
  
  \textbf{Discussion:}
    \begin{itemize}
        \item \colorbold{Morphology} and \colorbold{context} help predict useful embeddings.
        \item \colorbold{The attention mechanism works}: depending on the task, the network will use either more the context or the morphology to generate an embedding.
    \end{itemize}
    
    \textbf{Future works:}
    \begin{itemize}
        \item  Apply the \colorbold{attention mechanism on each character of the OOV word and each word of the context} instead of using the hidden state of the respective elements only.
        \item Test our attention model in \colorbold{different languages} and on other NLP tasks, such as \colorbold{machine translation}.
    \end{itemize}
    \vspace{-3.5mm}
  }
\end{columns}

\end{document}